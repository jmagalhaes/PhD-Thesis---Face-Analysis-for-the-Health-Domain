%%%%%%%%%%%%%%%%%%%%%%%%%%%%%%%%%%%%%%%%%%%%%%%%%%%%%%%%%%%%%%%%%%%%
%% abstrac-pt.tex
%% UNL thesis document file
%%
%% Abstract in Portuguese
%%%%%%%%%%%%%%%%%%%%%%%%%%%%%%%%%%%%%%%%%%%%%%%%%%%%%%%%%%%%%%%%%%%%
Comunicação é fundamental no nosso dia-a-dia. Através da comunicação trocamos idéias, opiniões, sentimentos e muito mais.
No entanto, problemas neurológicos ou lesões cerebrais traumáticas podem afetar a produção de fala através de anormalidades no controle motorico da fala.
Além de causar dificuldades na produção de fala, problemas neurológicos também podem causar dificuldade em expressar emoções. Pacientes com Parkinson, por exemplo, desenvolvem dificuldades na produção de fala e na manutenção da qualidade da voz, como também na expressividade emocional facial.
Embora a expressão de emoções não seja diretamente afetada por um problema neurológico existente, as incapacidades de fala podem causar
frustração e auto-estima reduzida que, por outro lado, pode piorar os sintomas de distúrbios da fala. Um excelente exemplo sobre este assunto é a gaguez.


À medida que a comunicação humana envolve o processamento complexo de diferentes fontes de informação, esta tese de doutorado visa desenvolver uma estrutura de \textit{multimodal machine learning} capaz de relacionar características de distúrbios de fala motora, como disartria, apraxia ou  gaguez, com emoções expressas. Para o efeito, serão exploradas técnicas de ponta utilizadas na análise de vídeo, áudio e dados fisiológicos para desenvolver uma representação multimodal que capture as correspondências entre as modalidades.


Ao desenvolver uma estrutura multimodal capaz de detectar expressões de emoções (afeto), bem como erros de fala em diferentes modalidades, a esperança é confirmar matematicamente a relação entre a gravidade dos distúrbios da fala motora e a expressividade emocional. Isso poderia ser uma nova medida de gravidade dos problemas neurodegenerativos e trazer novos conhecimentos para o desenho da terapia, por exemplo, fornecer aos pacientes biofeedback, de modo que eles possam adaptar seu estado emocional a um que suporte a produção de fala em vez de
sabota-lo.


% Palavras-chave do resumo em Inglês
\begin{keywords}
facial expressions, motor speech disorders, stuttering, dysarthria, apraxia of speech, emotional expressiveness, multimodal machine learning 
\end{keywords} 


