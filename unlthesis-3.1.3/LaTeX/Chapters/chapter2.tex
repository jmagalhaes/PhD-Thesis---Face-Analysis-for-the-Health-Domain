%%%%%%%%%%%%%%%%%%%%%%%%%%%%%%%%%%%%%%%%%%%%%%%%%%%%%%%%%%%%%%%%%%%%
%% chapter2.tex
%% UNL thesis document file
%%
%% Chapter with the template manual
%%%%%%%%%%%%%%%%%%%%%%%%%%%%%%%%%%%%%%%%%%%%%%%%%%%%%%%%%%%%%%%%%%%%
\chapter{Fundamentals}
\label{cha:fundamentals}

In this chapter the difference between verbal and non-verbal communication will be clarified. To understand what factors influence the quality of speech, in \ref{sec:anatomy} the anatomy of speech production will be explained.  



% ==================================================================
\section{Verbal vs. Non-verbal Communication} % (fold)
\label{sec:communication}


% ==================================================================
\section{Speech Production Anatomy} % (fold)
\label{sec:anatomy}


% ==================================================================
\section{Computational Representations of Human Faces}

\subsection{Facial action coding system (FACS)}




\subsection{Geometric landmarks}

\subsection{3D representations}

\subsection{Thermal}


% ==================================================================
\section{Speech Language Pathology}
\label{sec:SLP}
\cite{SLPathologies}: prevalence in the US on page 10

Diagnosis and treatment of:\\
- swallowing (dysphagia) \\
- speech-language disorders\\
- cognitive-communication disorders\\


Speech-Language Pathologists treat disorders of:\\
- speech sound production (e.g., articulation, apraxia, dysarthria)\\
- resonance (e.g., hypernasality, hyponasality)\\
- voice (e.g., phonation quality, pitch, respiration)\\
- fluency (e.g., stuttering)\\
- language (e.g., comprehension, expression, pragmatics, semantics, syntax)\\
- cognition(e.g., attention, memory, problem solving, executive functioning)\\
- feeding and swallowing (e.g., oral, pharyngeal, and esophageal stages) \\

Some etiologies:\\
- neonatal problems (e.g., prematurity, low birth weight, substance exposure)\\
- developmental disabilities (e.g., specific language impairment, autism spectrum
disorder, dyslexia, attention deficit/hyperactive disorder)\\
- oral anomalies (e.g., cleft lip/palate, dental malocclusion, macroglossia, oral-motor
dysfunction)\\
- neurological disease/dysfunction (e.g., traumatic brain injury, cerebral palsy, cerebral
vascular accident, dementia, Parkinson's disease, amyotrophic lateral sclerosis)\\



% ==================================================================
\begin{comment}
\section{Example glossary and acronyms}
%
% \todo[inline]{A a note in a line by itself.}
%
This is the first occurrence of an abbreviation: \gls{abbrev}.

And now the second occurrence of the same abbreviation: \gls{abbrev}.

And a new acronym with capital letter: \Gls{xpt} and reused \gls{xpt}.

Lets add the term ``\gls{computer}'' to the glossary!

\end{comment}