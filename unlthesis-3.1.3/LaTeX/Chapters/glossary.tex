%%%%%%%%%%%%%%%%%%%%%%%%%%%%%%%%%%%%%%%%%%%%%%%%%%%%%%%%%%%%%%%%%%%%
%% glossary.tex
%% UNL thesis document file
%%
%% Glossary definition
%%%%%%%%%%%%%%%%%%%%%%%%%%%%%%%%%%%%%%%%%%%%%%%%%%%%%%%%%%%%%%%%%%%%
\newglossaryentry{computer} {
	name={computer}, 
	description={An electronic device which is capable of receiving information (data) in a particular form and of performing a sequence of operations in accordance with a predetermined but variable set of procedural instructions (program) to produce a result in the form of information or signals.
}
}

\newglossaryentry{aug} {
	name={Action Unit}, 
	description={visible results from the contraction or relaxation of one or more muscles, used to describe higher level concepts in the Facial Action Coding System
}%,
%first={\glsentrylong{als}}(\glsentryshort{als}),
%	long={Amyotrophic Lateral Sclerosis},
%	firstplural={\glsentrylong{als}\glspluralsuffix\ (\glsentryname{als}\glspluralsuffix )}
}

\newglossaryentry{au}{
type=\acronymtype,
name={AU},
description={Action Unit, \glsseeformat[Glossary:]{aug}{}},
first={Action Unit (AU)\glsadd{aug}}
}

\newglossaryentry{comm_dis} {
	name={communication disorder}, 
	description={A communication disorder is an impairment in the ability to receive, send, process, and comprehend concepts of verbal, nonverbal, and graphic symbol systems.(\cite{SLPathologies} p.12)
}
}

\newglossaryentry{rehab_serv} {
	name={rehabilitative services}, 
	description={Rehabilitative services help restore or improve abilities lost or impaired as a result of illness, disease, injury, or disability.(\cite{SLPathologies} p.12)
}
}

\newglossaryentry{hab_serv} {
	name={habilitative services}, 
	description={Habilitative services and therapies are designed to develop new skills and maximize functioning, while rehabilitative services and therapies help a person recover skills that have been lost or impaired. (\cite{SLPathologies} p.12)
}
}

\newglossaryentry{develop_conditions} {
	name={developmental conditions}, 
	description={Developmental disorders, developmental disabilities, and developmental delays refer to specific impairments that differ from the normal condition. Stedman’s medical dictionary, 27th edition (2000), defines development as “the act or process of natural progression in physical and psychological maturation from previous, lower, or embryonic stage to a later, more complex, or adult stage.” Development is a natural state, but when paired with disorder, disability, or delay, it indicates an abnormal state.(\cite{SLPathologies} p.12)
}
}

\newglossaryentry{acc} {
	name={Augmentative and Alternative Communication}, 
	short={ACC},
	description={AAC system is any combination of devices, aids, techniques, symbols, and/or strategies to represent and/or augment spoken and/or written language or to provide an alternative mode of communication. (\cite{SLPathologies} p.16)
},
first={\glsentrylong{acc}}(\glsentryshort{acc}),
	long={Augmentative and Alternative Communication},
	firstplural={\glsentrylong{acc}\glspluralsuffix\ (\glsentryname{acc}\glspluralsuffix )}
}

\newglossaryentry{alsg} {
	name={Amyotrophic Lateral Sclerosis}, 
	%short={ALS},
	description={ALS is a neurological disease resulting in progressive muscle weakness and atrophy and is commonly called Lou Gehrig disease. (\cite{SLPathologies} p.33)
}%,
%first={\glsentrylong{als}}(\glsentryshort{als}),
%	long={Amyotrophic Lateral Sclerosis},
%	firstplural={\glsentrylong{als}\glspluralsuffix\ (\glsentryname{als}\glspluralsuffix )}
}

\newglossaryentry{als}{
type=\acronymtype,
name={ALS},
description={Amyotrophic Lateral Sclerosis, \glsseeformat[Glossary:]{alsg}{}},
first={Amyotrophic Lateral Sclerosis (ALS)\glsadd{alsg}}
}



\newglossaryentry{hdg} {
	name={Huntington's disease}, 
	%short={ALS},
	description={todo}
}

\newglossaryentry{hd}{
type=\acronymtype,
name={HD},
description={Huntington's disease, \glsseeformat[Glossary:]{hdg}{}},
first={Huntington's disease (HD)\glsadd{hdg}}
}




\newglossaryentry{adg} {
	name={Alzheimer's disease}, 
	%short={ALS},
	description={todo}
}

\newglossaryentry{ad}{
type=\acronymtype,
name={AD},
description={Alzheimer's disease, \glsseeformat[Glossary:]{adg}{}},
first={Alzheimer's disease (AD)\glsadd{adg}}
}






\newglossaryentry{aphasia} {
	name={aphasia}, 
	description={Aphasia is a language disorder that results from damage to portions of the brain that are responsible for language. All aspects of language (speaking, writing, reading, and understanding) may be affected to some degree. The common cause of aphasia is a cerebral vascular accident (CVA) or stroke. The disorder may impair the expression and understanding of language. Aphasia may co-occur with speech disorders such as dysarthria or apraxia of speech. The nature and severity of aphasia will vary from individual-to-individual as will the treatment plan and approaches used.(\cite{SLPathologies} p.33)
}
}


\newglossaryentry{apraxia} {
	name={Apraxia}, 
	description={Apraxia of speech is a motor speech disorder characterized by difficulty planning, sequencing, and organizing motor or muscle movements specifically for the production of speech. The patient may have trouble forming words or speaking despite the ability to use the oral and facial muscles to make sounds. Apraxia of speech is caused by damage to the parts of the brain
that control muscle movement. There are two main types of speech apraxia: acquired apraxia of speech and childhood apraxia of speech (CAS).(\cite{SLPathologies} p.34)
}
}

\newglossaryentry{aquired_apraxia} {
	name={aquired apraxia of speech}, 
	description={Acquired apraxia of speech can affect a person at any age, although it most typically occurs in adults. It is caused by damage to the parts of the brain that are involved in speaking and involves the loss or impairment of existing speech abilities. The disorder may result from a stroke, head injury, tumor, or other illness affecting the brain. Acquired apraxia of speech may
occur together with muscle weakness affecting speech production (dysarthria) or language difficulties caused by damage to the nervous system (aphasia).(\cite{SLPathologies} p.34)
}
}

\newglossaryentry{cas} {
	name={childhood apraxia of speech}, 
	short={CAS},
	description={Childhood apraxia of speech (CAS) is a neurological childhood (pediatric) speech sounds disorder in which the precision and consistency of movements underlying speech are impaired in the absence of neuromuscular deficits (e.g., abnormal reflexes, abnormal tone). CAS may occur as a result of known neurological impairment, in association with complex neurobehavioral disorders of known or unknown origin, or as an idiopathic neurogenic speech
sound disorder. The core impairment in planning and/or programming spatiotemporal parameters of movement sequences results in errors in speech sound production and prosody. CAS is not a developmental delay and a child will not outgrow this disorder. It is not an educational issue, but rather an issue of health and normal physiological function. Developmental delay describes a slower than normal rate of development, but CAS is a disorder.(\cite{SLPathologies} p.34)
},
first={\glsentrylong{cas}}(\glsentryshort{cas}),
	long={childhood apraxia of speech},
	firstplural={\glsentrylong{cas}\glspluralsuffix\ (\glsentryname{cas}\glspluralsuffix )}
}

\newglossaryentry{asd} {
	name={Autism Spectrum Disorder}, 
	short={ASD},
	description={Autism spectrum disorder (ASD) is a neurodevelopmental disorder that impairs an individual’s ability to process and integrate information; is characterized by speech, language, and communication impairments; and affects social and cognitive abilities. As the term "spectrum" indicates, there can be a wide range of effects. ASD includes Asperger disorder, pervasive developmental disorder, and Rett disorder.(\cite{SLPathologies} p.36)
},
first={\glsentrylong{asd}}(\glsentryshort{asd}),
	long={Autism Spectrum Disorder},
	firstplural={\glsentrylong{asd}\glspluralsuffix\ (\glsentryname{asd}\glspluralsuffix )}
}

\newglossaryentry{cpg} {
	name={Cerebral Palsy}, 
	%short={CP},
	description={Cerebral palsy (CP) is a movement disorder caused by damage to the brain before, during, or soon after birth. The ability of people with CP to communicate effectively is often impaired by problems with speech and gestures commonly used in communication. Communication difficulties associated with cerebral palsy can be multifactorial, arising from motor, intellectual, and/or sensory impairments, and children with this diagnosis can experience mild to severe difficulties in expressing themselves and in swallowing.They are often referred for speech and language therapy services to maximize their communication and swallowing skills and help them to take a role that is as independent as possible.(\cite{SLPathologies} p.37)
}%,
%first={\glsentrylong{cp}}(\glsentryshort{cp}),
%	long={Cerebral Palsy},
%	firstplural={\glsentrylong{cp}\glspluralsuffix\ (\glsentryname{cp}\glspluralsuffix )}
}

\newglossaryentry{cp}{
type=\acronymtype,
name={CP},
description={Cerebral Palsy, \glsseeformat[Glossary:]{cpg}{}},
first={Cerebral Palsy (CP)\glsadd{cpg}}
}

\newglossaryentry{cogn_comm_dis} {
	name={Cognitive-Communication Disorder}, 
	description={Communication requires a complex interplay between cognition, language, and speech. Cognitive-communication disorders are cognitive deficits that affect one or all of the following areas: attention (including visuospatial neglect), memory, problem solving, reasoning, organizing, and planning. These deficits impact communication by decreasing comprehension, expression, and pragmatics (the use and interpretation of verbal and nonverbal language in social interaction). People who have suffered a traumatic brain injury (TBI) frequently exhibit cognitive-communication disorders. Damage to the right hemisphere of the brain (RHD), often due to stroke, can result in deficits of cognition and communication. Other neurological events and diseases may result in cognitive-communication deficits, including but not limited to encephalopathy, multiple sclerosis, Parkinson’s disease, dementia, stroke, brain tumors. (\cite{SLPathologies} p.39)
}
}

\newglossaryentry{dementia} {
	name={dementia}, 
	description={Dementia is a syndrome resulting from acquired brain disease and characterized by progressive deterioration in memory and other cognitive domains (e.g., language, judgment, abstract thinking, and executive functioning). The most common cause of dementia is Alzheimer's disease. Declines in memory and other cognitive functions affect the ability to comprehend and produce linguistic information. Individuals with dementia may have difficulty following a conversation or following simple directions. Often they lose the topic, miss the point, and repeat themselves. Verbal output is reduced and is less substantive, and they are less efficient in expressing needs. Verbal output of late stage individuals appears nonsensical, and many late stage patients are unable to communicate even basic needs. Difficulty in swallowing may be seen. (\cite{SLPathologies} p.39)
}
}

\newglossaryentry{dysarthria} {
	name={Dysarthria}, 
	description={Dysarthria represents a group of motor speech disorders characterized by weakness, slowness, and/or lack of coordination of the speech musculature as the result of damage to the central or peripheral nervous system. Phonation, respiration, resonance, articulation, and prosody are affected. Movements may be impaired in force, timing, endurance, direction, and range of motion. Symptoms may include slurred speech, weak or imprecise articulatory contacts, weak respiratory support, low volume, incoordination of the respiratory stream, hypernasality, and reduced intelligibility. (\cite{SLPathologies} p.40)
}
}

\newglossaryentry{dysphagia} {
	name={Dysphagia}, 
	description={Dysphagia, or difficulty in swallowing, can result in choking, pulmonary problems, inadequate nutrition and hydration, and weight loss. It can cause food to enter the airway that may lead to complications including death from aspiration pneumonia. For infants, difficulties in sucking and breathing, in addition to swallowing, severely compromise nutrition. For children, lack of weight gain is like weight loss in adults. Children may refuse specific foods and textures because of reduced oral motor skills. Causes of dysphagia include TBI, stroke, central nervous system infection, head and neck cancer, effects of radiation, degenerative diseases, congenital conditions (e.g., cerebral palsy), anatomic and structural problems (e.g., cleft palate), and psychosocial and behavioral issues. (\cite{SLPathologies} p.41)
}
}

\newglossaryentry{msg} {
	name={Multiple Sclerosis}, 
	%short={MS},
	description={Rehabilitation is considered a necessary component of comprehensive, quality health care for people with multiple sclerosis (MS), at all stages of the disease (National Multiple Sclerosis Society, n.d.). People with MS often have swallowing difficulties as well as speech problems. Cognitive skills and memory can also be impaired. Dysarthria, in which speech patterns may be disrupted or words slurred, occurs in approximately 40\% of all patients with MS (Merson \& Rolnick, 1998). When speech and voice disturbances do occur, they usually present as a spasticataxic dysarthria with disorders of voice intensity, voice quality, articulation, and intonation. Treating dysarthria, dysphagia, and cognitive deficits in MS patients is effective for reestablishing functional daily activities. (\cite{SLPathologies} p.45)
}%,
%first={\glsentrylong{ms}}(\glsentryshort{ms}),
%	long={Multiple Sclerosis},
%	firstplural={\glsentrylong{ms}\glspluralsuffix\ (\glsentryname{ms}\glspluralsuffix )}
}

\newglossaryentry{ms}{
type=\acronymtype,
name={MS},
description={Multiple Sclerosis, \glsseeformat[Glossary:]{msg}{}},
first={Multiple Sclerosis (MS)\glsadd{msg}}
}

\newglossaryentry{neuro_motorspeech} {
	name={Neurological Motor Speech}, 
	description={Neurological motor speech assessment looks at the structure and function of the oral motor mechanism for non-speech and speech activities including assessment of muscle tone, muscle strength, motor steadiness and speech, range, and accuracy of motor movements. Speech characteristics of the phonatory-respiratory system (pitch, loudness, voice quality), resonance, articulation, and prosody are assessed. (\cite{SLPathologies} p.46)
}
}

\newglossaryentry{pdg} {
	name={Parkinson’s disease},
%	short={PD},
	description={Many people with Parkinson's disease suffer from disorders of speech and voice. Cognitive skills and memory can also be impaired. These disorders are typically characterized by speech and voice that are monotonous, quiet, hoarse, and breathy. People with Parkinson's disease also tend to give fewer non-verbal cues, such as facial expressions and hand gestures. These disabilities tend to increase as the disease progresses and can lead to serious problems with communication and swallowing. (\cite{SLPathologies} p.47)
},
%first={\glsentrylong{pd}}(\glsentryshort{pd}),
%	long={Parkinson’s disease}
	%firstplural={\glsentrylong{pd}\glspluralsuffix\ (\glsentryname{pd}\glspluralsuffix )}
}

\newglossaryentry{pd}{
type=\acronymtype,
name={PD},
description={Parkinson’s disease, \glsseeformat[Glossary:]{pdg}{}},
first={Parkinson’s disease (PD)\glsadd{pdg}},
%see=[Glossary:]{pdg}
}

\newglossaryentry{social_comm_dis} {
	name={Social Communication Disorder}, 
	description={Social communication disorders may include problems with social interaction, social cognition, and pragmatics (the social use of language). A social communication disorder may be a distinct diagnosis or it may exist with other conditions, such as autism, TBI, or intellectual disabilities. Social communication behaviors include eye contact, facial expressions, body language, and emotional expression. (\cite{SLPathologies} p.48)
}
}


\newglossaryentry{speech_sound} {
	name={Speech Sound Disorders}, 
	description={Speech sound impairments may arise from problems with articulation (making sounds) and phonological processes (sound patterns). Articulation disorders include problems with articulation and may involve sound substitutions, omissions, additions, or distortions. A phonological process disorder involves patterns of sound errors. For example, sounds made in the back of the mouth like "k" and "g" may be substituted for those in the front of the mouth like "t" and "d" (e.g., saying "tup" for "cup" or "das" for "gas"). (\cite{SLPathologies} p.48)
}
}


\newglossaryentry{stuttering} {
	name={Stuttering}, 
	description={Stuttering (stammering) is a speech disorder in which sounds, syllables, or words are repeated or prolonged, disrupting the normal flow of speech. These speech disruptions may be accompanied by struggling behaviors, such as rapid eye blinks or tremors of the lips. Stuttered speech often includes repetitions of words or parts of words, as well as prolongations of speech sounds. Speech may become completely stopped or blocked, so that the mouth is positioned to say a sound, sometimes for several seconds, with little or no sound forthcoming. After some effort, the person may complete the word. Interjections such as "um" or "like" can occur as well. In 2010, for the first time, National Institute on Deafness and Other Communication Disorders (NIDCD) researchers isolated three genes that cause stuttering. (\cite{SLPathologies} p.48)
}
}

\newglossaryentry{cluttering} {
	name={cluttering}, 
	description={Cluttering is a syndrome characterized by a speech delivery rate that is abnormally fast and/or irregular. Cluttered speech is characterized by one or more of the following: (1) failure to maintain normally expected sound, syllable, phrase, and pausing patterns and/or (2) greater than expected incidents of dysfluency, the majority of which are unlike those typical of people who stutter. Examples of cluttered speech include compressed consonant clusters, unfinished words, and shortened vowels. (\cite{SLPathologies} p.49)
}
}

\newglossaryentry{tbig} {
	name={Traumatic Brain Injury}, 
	%short={TBI},
	description={Traumatic brain injury (TBI) can result in lifelong impairment of physical, cognitive, and psychosocial functioning. Adults and children who have experienced a TBI frequently exhibit cognitive-communication disorders. Communication requires a complex interplay between cognition, language, and speech, with cognitive processes ranging from basic to complex and includes attention, memory, reasoning, and executive functions. Communication involves listening, reading, writing, speaking, and gesturing at all levels of language. Speech-language pathologists provide cognitive-communication treatment to TBI patients through evaluation of specific language and cognitive deficits and development of an intervention program. (\cite{SLPathologies} p.49)
}%,
%first={\glsentrylong{tbi}}(\glsentryshort{tbi}),
%	long={Traumatic Brain Injury},
%	firstplural={\glsentrylong{tbi}\glspluralsuffix\ (\glsentryname{tbi}\glspluralsuffix )}
}

\newglossaryentry{tbi}{
type=\acronymtype,
name={TBI},
description={Traumatic Brain Injury, \glsseeformat[Glossary:]{tbig}{}},
first={Parkinson’s disease (PD)\glsadd{tbig}}
}

\newglossaryentry{vpd} {
	name={Velopharyngeal Dysfunction}, 
	short={VPD},
	description={The purpose of the velopharyngeal mechanism is to close off the nasal cavity from the oral cavity during speech, normalizing both resonance and articulation for pressure sensitive phonemes. Closure is accomplished by the action of the velum, the lateral pharyngeal walls, and the posterior pharyngeal walls. Failure of these muscles to close during speech tasks results in velopharyngeal dysfunction (VPD). VPD also allows for the leakage of air into the nasal cavity, thus causing nasal resonance and reduced oral pressure. Resonance can be assessed as normal, hypernasal, hyponasal, or mixed hyper/hyponasal. The presence of hypernasality (too much sound resonating in the nasal cavity during speech, usually on vowels and voiced oral consonants) is often prevalent in this disorder. Audible nasal emission of air through the nasal cavity during oral pressure consonants may also be a finding. (\cite{SLPathologies} p.50)
},
%first={\glsentrylong{ms}}(\glsentryshort{vpd}),
%	long={Velopharyngeal Dysfunction},
%	firstplural={\glsentrylong{vpd}\glspluralsuffix\ (\glsentryname{vpd}\glspluralsuffix )}
}



%========= ACRONYMS ONLY===========
\newglossaryentry{slp}{
type=\acronymtype,
name={SLP},
description={Speech and Language Pathologist},
first={Speech and Language Pathologist (SLP)},
plural={SLPs},
descriptionplural={Speech and Language Pathologists},
firstplural={\glsentrydescplural{slp} (\glsentryplural{slp})}
}


\newglossaryentry{pwd}{
type=\acronymtype,
name={PWD},
description={people who stutter},
first={people who stutter (PWD)},
}

\newglossaryentry{facs}{
type=\acronymtype,
name={FACS},
description={Facial Action Coding System},
first={Facial Action Coding System (FACS)},
}

\newglossaryentry{eeg}{
type=\acronymtype,
name={EEG},
description={Electroencephalogram},
first={Electroencephalogram (EEG)},
}

\newglossaryentry{eda}{
type=\acronymtype,
name={EDA},
description={Electrodermal activity},
first={Electrodermal activity (EDA)},
}