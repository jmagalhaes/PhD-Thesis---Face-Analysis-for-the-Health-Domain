%%%%%%%%%%%%%%%%%%%%%%%%%%%%%%%%%%%%%%%%%%%%%%%%%%%%%%%%%%%%%%%%%%%%
%% chapter5.tex
%% UNL thesis document file
%%
%% Chapter with lots of dummy text
%%%%%%%%%%%%%%%%%%%%%%%%%%%%%%%%%%%%%%%%%%%%%%%%%%%%%%%%%%%%%%%%%%%%
\chapter{Research Plan}
\label{cha:plan}

4 pages

\section{Tasks}

Link this to a Gantt chart.

\textbf{Task 1- Research image features for RGB, depth, and infrared data that permit the mapping of visible facial muscle activity to correct/incorrect execution of speech exercises OR paralysis severity.} It is expected that by combining different modalities (RGB, depth, infrared) the feature representation of facial activity is improved. The goal is that the feature representation permits the binary classification of correct/incorrect execution of speech exercises in the case of applying to speech therapy. In the case of facial paralysis the goal is to have a feature representation that permits a multiclass classification for the degree of severity of the paralysis.\\

\textbf{Task 2 - Understand the relation ... engagement during exercises through detecting relevant facial expressions.} Speech therapy requires monotonous repetition of exercises which can affect the engagement of the patient. By detecting in real-time the disengagement of the patient, different exercises can be suggested to maintain a positive learning curve. \\

\textbf{Task 3 - Research feature spaces that relates in meaningfully manner information from different modalities, therapist annotations and time.}
Suuport the SLTs reasoning and exploration of health data. 
Additionally to visual features, therapist annotations will be used to represent the progress of the therapy over time. As the therapy is performed through several sessions the time component is essential to measure the progress of a patient. 

\textbf{Task 4 - Develop statistical model that suggests future exercises by comparing one patient with similar ones in an existing database.} By being able to model the temporal progress of the therapy, the progress of one patient can be compared to the progress of others. Patients with similar starting point and positive conclusion of the therapy, can be taken as example and can provide insights of therapeutic actions that can help other similar patients to improve their progress. Thus, future actions to take can be suggested to the therapist. This model could also be used for other applications where temporal progression of a disease is observed through several therapy sessions. 
\section{Plan for 3 chapters of work}