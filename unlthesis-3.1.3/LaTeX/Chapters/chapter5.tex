%%%%%%%%%%%%%%%%%%%%%%%%%%%%%%%%%%%%%%%%%%%%%%%%%%%%%%%%%%%%%%%%%%%%
%% chapter4.tex
%% UNL thesis document file
%%
%% Chapter with lots of dummy text
%%%%%%%%%%%%%%%%%%%%%%%%%%%%%%%%%%%%%%%%%%%%%%%%%%%%%%%%%%%%%%%%%%%%
\chapter{Current Work and Preliminary Results}
\label{cha:currentWork}

\section{Clinical data}

4 pages

\subsection{BVS data}

\subsection{Physionet datasets}


https://physionet.org/mimic2/
https://github.com/RigautAntoine/deep-computational-phenotyping-papers

\subsection{FERA 2017}


\subsection{USC-TIMIT: A MULTIMODAL ARTICULATORY DATA CORPUS FOR SPEECH RESEARCH}

A Multimodal Real-Time MRI Articulatory Corpus for Speech Research.
\begin{itemize}
    \item 10 American English talkers (5M, 5F).
    \item Real Fme MRI (5 speakers also with EMA)
and synchronized audio.
    \item 460 sentences each (>20 minutes)
    \item Freely available for speech research.
\end{itemize}


\url{http://sail.usc.edu/span/usc-timit}

from \url{http://people.csail.mit.edu/jrg/meetings/2016-Feb8-ShriN.pdf}\\

Seeking a window into the human mental state through engineering approaches\\

German Audio-Visual Spontaneous Speech Database:
\url{http://emotion-research.net/download/vam}


\section{Inital benchmark}

\begin{itemize}
\item get insight about data and methods
\item be critical about the results
\end{itemize}



\subsection{support your proposed research direction with the critical analysis}
the goal of this section is to help you in making an informed and mature discussion with your PhD advisory committee.


