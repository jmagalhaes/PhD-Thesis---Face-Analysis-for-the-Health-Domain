\chapter{Methodology}
\label{cha:methods}



Anticipate the methods you will use to achieve the project aims and show that project is feasible in the time period.

\section{Data}

\subsection{Available Datasets}

TORGO Database of Dysarthric Articulation

UA-Speech Database

AVEC 2016 multimodal dataset \url{http://sspnet.eu/avec2016/challenge-guidelines/}, \url{http://sspnet.eu/avec2016/}
Depression subchallenge: SimSensei corpus of human-agent interactions\\
Emotion subchallenge: RECOLA multimodal corpus of remote and collaborative affective interactions (audio-visual-physiological emotion recognition.)\\

AVEC 2017: \\
Emotion Sub-Challenge data: SEWA \url{https://db.sewaproject.eu}\\
Depression Sub-Challenge data: DAIC-WOZ Depression Database \url{http://dcapswoz.ict.usc.edu/wwwutil_files/DAICWOZDepression_Documentation.pdf}

\subsection{Creation of own Dataset}
- Fieldwork required: where, how long and at what intervals?


\section{Toolboxes}


\section{Hardware}





\section{Expertise Consulting}

Stuttering Center of Western Pennsylvania is a collaboration between the Department of Audiology and Speech-Language Pathology at Children's Hospital of Pittsburgh of UPMC and the Department of Communication Science and Disorders at the University of Pittsburgh. Contact Dr. J. Scott Yaruss, Director of the Stuttering Center,jsyaruss@pitt.edu.\\


University of Pittsburgh Voice Center : Rita Hersan, MS, CCC-SLP. She is a certified clinician for Lee Silverman Voice Treatment, an innovative and clinically-proven method for improving voice and speech in individuals with Parkinson disease. Ms. Hersan is also the co-author of Adventures in Voice, a specific program for children with voice disorders.\url{http://www.surgery.pitt.edu/centers-excellence/voice-center/meet-voice-center}
\\

Department of Neurology, University of Pittsburgh.
\url{http://www.neurology.upmc.edu/movement/}



Associations:
\url{https://www.apdaparkinson.org}


\section{Ethics Commission}

\section{Possible Difficulties}